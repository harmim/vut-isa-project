% Author: Dominik Harmim <xharmi00@stud.fit.vutbr.cz>


\documentclass[a4paper, 11pt]{article}


\usepackage[czech]{babel}
\usepackage[utf8]{inputenc}
\usepackage[left=2cm, top=3cm, text={17cm, 24cm}]{geometry}
\usepackage{times}
\usepackage{graphicx}
\usepackage[unicode]{hyperref}
\hypersetup{
	colorlinks = true,
	hypertexnames = false,
	citecolor = red
}


\begin{document}
	%%%%%%%%%%%%%%%%%%%%%%%%%%%%%%%% Titulní stránka %%%%%%%%%%%%%%%%%%%%%%%%%%%
	\begin{titlepage}
		\begin{center}
			\includegraphics[width=0.77\linewidth]{inc/FIT_logo.pdf} \\

			\vspace{\stretch{0.382}}

			\Huge{Projektová dokumentace} \\
			\LARGE{\textbf{
				Čtečka novinek ve formátu Atom a~RSS s~podporou TLS
			}} \\
			\Large{Síťové aplikace a~správa sítí}

			\vspace{\stretch{0.618}}
		\end{center}

		{\Large
			\today
			\hfill
			Dominik Harmim (xharmi00)
		}
	\end{titlepage}



	%%%%%%%%%%%%%%%%%%%%%%%%%%%%%%%% Obsah %%%%%%%%%%%%%%%%%%%%%%%%%%%%%%%%%%%%%
	\pagenumbering{roman}
	\setcounter{page}{1}
	\tableofcontents
	\clearpage



	%%%%%%%%%%%%%%%%%%%%%%%%%%%%%%%% Úvod %%%%%%%%%%%%%%%%%%%%%%%%%%%%%%%%%%%%%%
	\pagenumbering{arabic}
	\setcounter{page}{1}

	\section{Úvod}

	Cílem projektu bylo vytvořit komunikující aplikaci v~jazyce C/C++.
	Vytvořil jsem program, který stahuje XML soubory (tzv. \texttt{feedy})
	přes síť Internet a~tyto soubory zpracovává a~vypisuje v~nich uvedné
	informace.

	Program zpracovává XML soubory ve formátu \texttt{Atom} a~\texttt{RSS},
	přičemž podporované verze jsou \texttt{Atom 1.0} \cite{atom},
	\texttt{RSS 1.0} \cite{rss1} a~\texttt{RSS 2.0} \cite{rss2}.

	Program po spuštění stáhne zdroje a~na standardní výstup vypíše informace
	požadované uživatelem. Vstupní zdroje i~výstupní informace je možné
	nastavovat vstupními argumenty programu, viz \ref{section:usage}.



	%%%%%%%%%%%%%%%%%%%%%%%%%%%%%%%% Návrh a implementace %%%%%%%%%%%%%%%%%%%%%%
	\section{Návrh a~implementace}

	Aplikace je rozdělena do několika zdrojových souborů a~tříd, které
	implementují požadované chování. Veškeré zdrojové soubory se nachází
	v~adresáři \texttt{./src}. V~této kapitole jsou stručně popsány jednotlivé
	části implementace.


	\subsection{Vstupní bod programu, \texttt{feedreader.cpp}}

	V~tomto souboru se nachází funkce \texttt{main}, která příjme vstupní
	argumenty programu a~následně jsou zpracovány třídou
	\texttt{ArgumentProcessor} \ref{section:argument_processor}.
	Pokud zpracování argumentů selže, je program ukončen s~chybovým kódem.

	Dále je vytvořen seznam URL zdrojů, které se budou zpracovávat. Tento
	seznam je vytvořen třídou
	\\ \texttt{UrlListFactory} \ref{section:url_list_factory}.

	Seznam zdrojů URL je dále zpracováván třídou \texttt{FeedProcessor}
	\ref{section:feed_processor}. Pokud zpracování selže, program skončí
	s~chybovým kódem, v~opačném případě program úspěšně skončí.


	\subsection{Zpracování argumentů, \texttt{ArgumentProcessor}}
	\label{section:argument_processor}

	Třída \texttt{ArgumentProcessor} je definovaná v~souboru
	\texttt{ArgumentProcessor.hpp} a~implementovaná v~souboru
	\texttt{ArgumentProcessor.cpp}.

	Tato třída zpracovává vstupní argumenty programu programem \texttt{getopt}
	\cite{getopt}. Pro zpracování argumentů jsem použil funkci
	\texttt{getopt\_long}, která umí pracovat i~s~dlouhými jmény argumentů
	a~chová se na všech systémech ekvivalentně na rozdíl od funkce
	\texttt{getopt}.

	Při chybně zadaných argumentech je výpsána nápověda programu, případně
	chybová zpráva.


	\subsection{Zpracování chyb, \texttt{error.hpp}}

	V~tomto hlavičkovém souboru jsou definovaná makra pro výpis chybových
	zpráv na standardní chybový výstup.


	\subsection{Vytvoření seznamu URL zdrojů, \texttt{UrlListFactory}}
	\label{section:url_list_factory}

	Třída \texttt{UrlListFactory} je definovaná v~souboru
	\texttt{UrlListFactory.hpp} a~implementovaná v~souboru
	\texttt{UrlListFactory.cpp}.

	Tato třída vytváří seznam URL zdrojů buď ze zadaného URL zdroje jako
	argumentu programu nebo ze souboru \texttt{feedfile} zadaného jako
	argument programu, který obsahuje jednotlivé URL zdrojů, na každém
	řádku jedno URL.

	Pokd zpracování souboru \texttt{feedfile} selže, je vypsaná chybová
	zpráva. Prázdné řádky a~řádky začínající znakem \texttt{\#} v~souboru
	\texttt{feedfile} jsou ignorovány. Počítá se s~tím, že každý řádek
	tohoto souboru bude zakončen znakem \texttt{LF}.


	\subsection{Zpracování zdrojů, \texttt{FeedProcessor}}
	\label{section:feed_processor}

	Třída \texttt{FeedProcessor} je definovaná v~souboru
	\texttt{FeedProcessor.hpp} a~implementovaná v~souboru
	\\ \texttt{FeedProcessor.cpp}.

	Tato třída stahuje jednotlivé zdrojové soubory a~následně je dále
	zpracovává. Stahování probíhá pomocí knihovny \texttt{OpenSSL}
	\cite{openSSL}. Při práci na tomto úkolu jsem studoval tento článek
	\cite{openSSL_tutorial}, a~proto jsem se při implementaci této třídy řídil
	některými doporučeními z~tohoto článku.

	Zpracovávám postupně jednotlivé URL zdrojů, jedno po druhém. Pokud je na
	vstupu zadaných více URL zdrojů a~zpracování jednoho ze drojů selže, tak
	pokračuji zpracováváním dalšího zdroje.

	Nejdříve provedu validaci a~prsování URL třídou \texttt{UrlParser}
	\ref{section:url_parser}. Pokud toto selže, tak vypisuji chybovou
	zprávu.

	Dále provádím připojení k~danému zdroji. Pokud se používá SSL, tak
	nastavím umístění certifikátů pro ověření platnosti certifikátů
	předložených serverem. Umístění těchto certifikátů nastavuji buď
	z~argumentů programu, pokud jsou tyto zadány nebo použiji výchozí
	umístění, které definuje \texttt{OpenSSL} knihovna. Pokud ověření
	certifikátů nebo připojení k~serveru selže, tak vypíši chybovou zprávu.

	Pokud úspěšně provede připojení k~serveru, tak se nejdříve odešle HTTP
	GET požadavek na daný zdroj daného serveru. Následně se přečtě HTTP
	odpověď serveru. Pokud při zápisu, či čtení HTTP požadavku, či odpovědi
	nastane chyba, tak se vypíše chybová zpráva.

	Při úspěšném přečtení HTTP odpovědi se provede parsování této odpovědi.
	Oddělí se hlavička od těla odpovědi a~provede se validace. Za validní
	odpovědi jsou považovány všechny odpovědi s~HTTP kódem s~hodnotou $ x $,
	kde $ 200 \leq x < 300 $. Pokud validace HTTP odpovědi selže, je vypsána
	chybová zpráva.

	Nakonec je provedeno parosvání XML dokumentu v~těle HTTP odpovědi
	a~výpis patřičných inforamcí třídou \texttt{XmlParser}
	\ref{section:xml_parser}. Pokud toto parsování selže, je vypsána chybový
	zpráva.


	\subsection{Parsování URL, \texttt{UrlParser}}
	\label{section:url_parser}

	Třída \texttt{UrlParser} je definovaná v~souboru
	\texttt{UrlParser.hpp} a~implementovaná v~souboru
	\texttt{UrlParser.cpp}.

	Tato třída slouží k~parsování URL. Tj. k~zjištění, jestli je předložené
	URL validní URL podle \texttt{RFC 3986} \cite{url} a~rozložení daného
	URL na jednotlivé části. Podporované jsou pouze schémata \texttt{http}
	a~\texttt{https}. Pokud v~URL není uveden port, tak se doplňí podle
	schématu, buď \texttt{80} nebo \texttt{443}.

	Samotné parsování URL probíhá pomocí knihovny pro regulární výrazy, viz
	\cite{regex}.


	\subsection{Parsování XML, \texttt{XmlParser}}
	\label{section:xml_parser}

	Třída \texttt{XmlParser} je definovaná v~souboru
	\texttt{XmlParser.hpp} a~implementovaná v~souboru
	\texttt{XmlParser.cpp}.

	Tato třída slouží k~parsování XML zdroje. Jsou podporovány pouze zdroje
	ve formátu \texttt{Atom 1.0} \cite{atom}, \texttt{RSS 1.0} \cite{rss1}
	a~\texttt{RSS 2.0} \cite{rss2}.

	Samotné parsování XML probíhá pomocí knihovny \texttt{Libxml2}
	\cite{libxml}, \cite{libxml_tutorial}.

	Pokud XML struktura parsovaného souboru není validní, v~souboru se
	nenachází kořenový element nebo typ XML souboru není podporovaný,
	vypíše se chybová zpráva.

	\subsubsection{Vypisování inforamcí ze zdrojového souboru}

	Nejdříve je vždy vypsán titulek daného zdrojového souboru. Pokud titulek
	ve zdroji chybí, tak se zpracovávání daného zdroje ukončí. Výpis titulku
	je vždy uvozen znaky \uv{*** } a~zanončen znaky \uv{ ***}.

	Následně jsou vypisovány jednotlivé položky zdrojových soburů. Nejdříve je
	vypsán titulek položky a~potom případně další doplňující informace.
	Nejdříve čas vytvoření nebo změny záznamu uvozen znaky \uv{Time: }, potom
	jméno, či e-mailová adresa autora dané položky uvozena znaky \uv{Author: }
	a~nakonec asociované URL uvozené znaky \uv{URL: }. Pokud některá
	z~položek není ve zdroji uvedena, tak se nebude vypisovat.


	\subsection{Překladový systém}
	\label{section:compilation}

	Překladový systém jako takový je vytvořen v~programu \texttt{CMake}.
	Definice tohoto systému se nachází v~souboru \texttt{CMakeLists.txt}.
	Nachází se zde definice názvu spustitelného souboru, definice zdrojových
	souborů, definice použitého C++ standardu, definice argumentů překladače
	a~definice linkovaných knihoven.

	Spouštení překladu je realizováno programem \texttt{GNU Make}, který po
	spuštění \texttt{make} nebo \texttt{make feedreader} přeloží program
	nástrojem \texttt{CMake} a~vytvoří spustitelný soubor \texttt{feedreader}
	v~kořenovém adresáři projektu. Spuštěním příkazu \texttt{make clean}
	je možné odstranit všechny soubory vytvoření při překladu.



	%%%%%%%%%%%%%%%%%%%%%%%%%%%%%%%% Návod na použití %%%%%%%%%%%%%%%%%%%%%%%%%%
	\section{Návod na použití}
	\label{section:usage}

	Po překladu programu, viz \ref{section:compilation}, je vytvořen
	spustitelný program \texttt{feedreader}.


	\subsection{Syntaxe a~sémantika spouštění programu}

	\texttt{
		./feedreader <url | -f <feedFile>> [-c <certFile>] [-C <certDir>]
			[-T] [-a] [-u]
	}
	\begin{itemize}
		\item \texttt{url} URL zdroje.
		\item \texttt{-f <feedFile>} Soubor \texttt{feedfile}. (Textový soubor,
			kde je na každém řádku uvedena URL zdroje. Prázdné řádky a~řádky
			začínající znakem \texttt{\#} jsou ignorovány. Poslední znak na
			každém řádku musí být \texttt{LF}.)
		\item \texttt{-c <certFile>} Soubor s~certifikáty pro ověření
			platnosti certifikátu předloženého serverem při použití SSL/TLS.
		\item \texttt{-C <certDir>} Adresář, ve kterém se vyhledávají
			certifikáty, které se použijí pro ověření platnosti certifikátu
			předloženého serverem při použití SSL/TLS.
		\item \texttt{-T} Pro každý zdroj se navíc zobrazí informace o~čase
			změny, či vytvoření zázamu.
		\item \texttt{-a} Pro každý zdroj se navíc zobrazí jméno, či e-mailová
			adresa autora záznamu.
		\item \texttt{-u} Pro každý zdroj se navíc zobrazí asociované URL
			záznamu.
	\end{itemize}
	Povinně je potřeba uvést buď URL zdroje nebo soubor \texttt{feedfile}.
	Podporovaná schémata zdrojů jsou \texttt{http} a~\texttt{https}.
	Parametry je možné zadávat v~libovolném pořadí. Je možné zadávat parametry
	bez argumentů i~následujícím způsobem \texttt{-Tau}. Protože je
	zpracování argumentů realizováno programem \texttt{getopt}, tak platí
	standardní chování definováné tímto programem, viz \cite{getopt}.

	Při chybně zadaných parametrech se vypíše napověda programu. Nápovědu
	je možné si vypsat i~příkazem \texttt{./feedreader -h} nebo
	\texttt{./feedreader --help}.



	%%%%%%%%%%%%%%%%%%%%%%%%%%%%%%%% Závěr %%%%%%%%%%%%%%%%%%%%%%%%%%%%%%%%%%%%%
	\section{Závěr}

	Podařilo se mi implementovat všechny části aplikace, které byly uvedené
	v~zadání projektu. Při konečném testování projektu jsem nezaznamenal žádné
	chybné chování programu.

	Projekt pro mě nebyl nijak časově náročný. Nastudování potřebných informací
	a~samotná implementace mi nezabrala více než několik hodin.

	Při práci na projektu jsem se především zdokonalil v programování
	v~jazyce C++, naučil jsem se pracovat s~knihovnou \texttt{OpenSSL}
	\cite{openSSL}, zjistil, jak funguje komunikace přes HTTPS a~jak funguje
	ověřování certifikátů \cite{openSSL_tutorial}, naučil jsem se pracovat
	s~knihovnou \texttt{Libxml2} \cite{libxml}, \cite{libxml_tutorial},
	nastudoval jsem si, jak je definován formát URI podle \texttt{RFC 3986}
	\cite{url}, naučil jsem se pracovat s~knihovnou pro práci s~regulárními
	výrazy \cite{regex}, naučil jsem se zpracovávat argumenty programu
	programem \texttt{getopt} \cite{getopt} a~nastudoval jsem si definici
	\texttt{Atom 1.0} \cite{atom}, \texttt{RSS 1.0} \cite{rss1}
	a~\texttt{RSS 2.0} \cite{rss2}.



	%%%%%%%%%%%%%%%%%%%%%%%%%%%%%%%% Citace %%%%%%%%%%%%%%%%%%%%%%%%%%%%%%%%%%%%
	\clearpage
	\bibliographystyle{czechiso}
	\renewcommand{\refname}{Literatura}
	\bibliography{manual}
\end{document}
